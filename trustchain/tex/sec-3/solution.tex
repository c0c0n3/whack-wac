\subsection{Proposed Solution}

wada wada

\subsubsection{Formal specification}
The formalisation and extension of the WAC specification will be
attained through subsequent refinement phases. In the tradition of
functorial semantics\cite{lawvere1963functorial, bonchi2017functorial},
a preliminary algebraic model will provide a concise, consistent and
unambiguous interpretation of WAC in terms of basic structures and
structure-preserving transformations in the elementary theory of the
categories of sets\cite{lawvere1964elementary, leinster2014rethinking}.
We will then engage with the WAC authors on GitHub to validate and
refine the model. Following that, we will devise a specification extension
to make the evaluation model more flexible so as to allow the evaluation
of access control policies expressed in terms of predicates on a generic
set---i.e., functions from a set $X$ to the Boolean algebra $\{0, 1\}$.
At the same time, we will investigate another specification extension
to accommodate data sharing through functional and homomorphic encryption
techniques. Finally, we will encode the formal specification in an
advanced programming language (either Haskell\cite{peytonjones:h98}
or Idris\cite{brady2013idris}) which we will then leverage to produce
a machine-checked proof of correctness of the resulting computer program.
The resulting (correct!) executable specification will be submitted to
the Solid project for consideration as a future version of WAC.

\begin{lstlisting}[language=Haskell]
data Tree a = Node a [Tree a]

path :: (a -> Bool) -> Tree a -> [a]
path p = collect []
  where
  collect xs (Node a ts) | p a       = a:xs
                         | otherwise = concatMap (collect (a:xs)) ts
\end{lstlisting}