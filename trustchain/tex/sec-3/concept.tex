\subsection{Concept and Objectives}

The past two decades have witnessed the rise of online, user-centred
services, most notably social media, which has resulted in a staggering
amount of user-generated content being produced and stored online.
However, individuals have seldom retained true ownership and control
of their own online data. Instead, few large corporations have been
amassing disproportionate amounts of user data, leaving individuals
with little to no control over how their data are shared, augmented
and processed to generate revenue. This state of affairs has raised
serious concerns about online data, ranging from the manifest, such
as security and privacy, to the subtle and hotly debated ethical and
societal implications.\cite{stremlau2018world}

Can individuals regain control of their own online data? Although
it appears to be a problem of gargantuan proportions, online data
privacy and security can and must be improved. Indeed, the inventor
of the Web himself, Tim Berners-Lee, has embarked on a journey to
address this very problem with the Solid project\cite{sambra2016solid}%
(https://solidproject.org/) under the auspices of the W3C. Solid aims
to repair the Web by enabling true data ownership and strengthened
privacy through distributed, decentralised applications controlled
by individuals rather than large corporations.

We intend to contribute to the Solid journey by improving and extending
Web Access Control (WAC), a core component of Solid's privacy and
security architecture. For individuals to truly regain control of
their online data in a decentralised architecture, strong correctness
guarantees are needed about the software implementing the WAC specification.
(Given that software is plagued with defects, why should a user trust
an implementation to be secure?) However, to ascertain whether an
implementation is correct, the specification must be unambiguous,
i.e., exactly one interpretation exists, and consistent, i.e., free
of contradictions.

Undoubtedly a remarkable achievement, the WAC specification nonetheless,
by dint of being expressed in plain English, suffers from ambiguity
and consistency issues. Such issues may jeopardise privacy and security
in a decentralised scenario where end-users share and/or migrate data
across servers and/or providers. To see why that may be the case,
consider WAC's resource hierarchies and access control policies (ACL).
The specification is not clear about exactly what constitutes a hierarchy
and states that a resource inherits its ACL policy from its container.
Now a conceivable resource hierarchy could be given by a resource $r$
owned by Alice and shared among two different containers $c_1$ and
$c_2$. For the sake of argument, assume policy $p_1$ is attached to
$c_1$, stating that user $u$ can read. Similarly a policy $p_2$ is
attached to $c_2$ and allows $u$ to delete. Futhermore, suppose two
server implementations exist $s_1$ and $s_2$ each with a different
interpretation of how $r$ should inherit its policy from the parent
container: $s_1$ selects $c_1$ as a parent whereas $s_2$ chooses $c_2$.
Alice keeps $r$ on $s_1$ and knows from experience with how $s_1$
enforces security that $u$ is not able to delete $r$. However after
migrating $r$ to $s_2$, to Alice's dismay, $u$ deletes $r$.

This proposal aims to achieve the following objectives.
\begin{description}
  \item[Formal specification] Make the WAC specification unambiguous
    and consistent so to be able to produce implementations that can
    be trusted to be secure and correct; Extend the specification to
    cater to data sharing through advanced cryptography which allows
    to selectively disclose only some parts of the data or have algorithms
    process encrypted data without revealing any actual content.
  \item[Access control policies] A domain-specific language to express
    access control rules in a language close to plain English.
  \item[Decentralised, trusted access control] Decentralised network
    to empower end-users to share and migrate their data and policies
    among servers without the risk of data breaches due to different
    WAC implementations interpreting the specification differently.
\end{description}
